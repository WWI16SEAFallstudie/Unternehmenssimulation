\clearpage
\chapter{Fazit/Ausblick}
Grundsätzlich ist die Aufgabe eine nette und fordernde Abwechslung zum normalen DHBW-Alltag. Das Planen und Entwickeln der verschiedenen Bestandteile hat Spaß gemacht und jeden in seiner Entwicklung und auf seine Weise weitergebracht.\\
Wir konnten uns recht schnell auf unsere "Industrie" einigen und hatten so mehr Zeit für die Planung und Entwicklung. Die meiste Zeit wurde für die Konzeption und Entwicklung der einzelnen Bestandteile verwendet. Es gab einige Diskussionen rund um die einzelnen Unternehmensabteilungen und deren Methoden. Grundsätzlich kamen wir aber immer auf einen Nenner im Sinne der Aufgabe.\\
Negativ war leider die kurze Zeit, die wir insgesamt hatten. Einige Ideen konnten so nur teilweise oder gar nicht umgesetzt werden. Auch das Schreiben einer recht ausführlichen Dokumentation hat uns einige Zeit gekostet zumal die Klausuren für dieses Semester ebenfalls an standen.\\   
Zusammenfassend kann man sagen, dass die Fallstudie ein guter Ansatz für einen realen Bezug ist, aber der enge Zeitraum einiges verhindert und somit rückblickend eher negativ bewertet werden muss.  