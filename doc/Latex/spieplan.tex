\clearpage
\chapter{Spielplan}
Spielplan "Watch Tycoon" \\
Anzahl der Spieler: 1-4 \\
Spielverfahren: Rundenbasiertes Strategiespiel mit Hot-Seat Verfahren \\
Maximal Anzahl der Spielrunden (insgesamt): 20 Runden \\
Zeitlimit pro Spielzug: 10 Minuten \\
\\
Jeder Spieler hat zu Beginn ein Grundkapital von HUNDERTTAUSEND EURO. Zu Beginn des Spiels muss eine Uhr aus den Segmenten Masse, Öko oder Luxus ausgewählt werden. Je nach Auswahl stehen verschiedene Ausstattungsmerkmale zur Verfügung. Der Spieler hat zehn Minuten Zeit um sein Spielzug abzuschließen. Sofern er das in der vorgegebene Zeit nicht erledigt hat, wird die Runde automatisch beendet und der nächste Spieler ist an der Reihe. Wenn alle Spieler an der Reihe waren, werden die gemachten Änderungen auf den fiktiven Markt übertragen. In der darauffolgenden Runde sind die Marktveränderungen dann in einem extra Menüpunkt einsehbar. \\ 
\\
Einkauf:\\
Im Bereich Einkauf kann der Rohstoff für die Produktion eingekauft werden. Dabei kann durch die Einkaufsmenge Rabatte erzielt werden. \\
\\
Verkauf: \\
Hier lässt sich die Verkaufsstrategie durch drei Distributionswege bestimmen. Diese wirken sich unterschiedlich auf die Verkaufszahlen aus.\\
\\
Marketing:\\
Im Marketing dreht sich alles um das Bekanntmachen einer neuen Uhr. Hierfür stehen fünf Werbemöglichkeiten zur Verfügung. Die verschiedenen Werbemaßnahmen wirken sich unterschiedlich auf den Markt und die Käuferschaft aus. Aber Achtung: Eine Werbekampagne muss sich nicht immer gut auswirken!\\ 
\\
Produktion: \\
In der Produktion können durch Zukäufe von Produktionsstraßen (maximal zwei weitere Produktionsstraßen möglich) entweder weitere Segmente eröffnet werden oder eine Kostensenkung durch ein bereits vorhandenes Segment erzielt werden. Des Weiteren können für die vorhandenen Produktionsstraßen Erweiterungen gekauft werden (maximal drei weitere möglich) um ebenfalls die Kosten zu senken und die Anfälligkeit für Defekte und Reparaturen zu senken. Bei dem Kauf einer Erweiterung wird die Anfälligkeit wieder auf null gesetzt. \\ 
\\
Forschung\&Entwicklung:\\
Die F\&E (Forschung\&Entwicklung) ist für die Erforschung von besseren Ausstattungsmerkmalen der jeweiligen Segmente zuständig. Durch das Erforschen der zusätzlichen Merkmale kann sich das Interesse der jeweiligen Uhr am Markt erhöhen.\\
\\
Die Unternehmensabteilungen Marketing, Verkauf, Einkauf, Produktion und F\&E sind ab Runde 1 verfügbar.\\
\\
Der Markt dient als Informationsinstrument für jeden Spieler. Dabei kann der Spieler die Nachfrage und das Marktvolumen einsehen und gegen Geld einmalig den Markt analysieren lassen. Sofern sich ein Spieler für die Produktion der Öko-Uhr entscheidet, erhält dieser zusätzlich einen Bonus, der später in der Endwertung mit einfließt.\\ 
\\
Ziel des Spiels/Gewinnbedingung:\\
Ziel des Spiels ist es nach Ablauf der 20 Spielrunden den größten Umsatz erzielt zu haben.


