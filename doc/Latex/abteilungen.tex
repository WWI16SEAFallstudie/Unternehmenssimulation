\clearpage
\chapter{Abteilungen}
\section{Produktion}
Die Abteilung Produktion stellt für den Spieler die wichtigsten Funktionen zur Herstellung seiner Uhr zu Verfügung dar.\\ 
Der Spieler hat in dieser Abteilung zwei Möglichkeiten seine Produktionsstraßen auszubauen und seine Produktion um die zwei weiteren Segmente zu erweitern:
\begin{enumerate}
	\item Erweiterung der Produktion
\begin{itemize}
	\item In der Produktion können maximal zwei weitere Produktionsstraßen hinzugekauft werden. Dabei kann für jedes Segment (Masse, Öko, Luxus) genau eine Produktionsstraße eröffnet werden. Durch ist das Unternehmen in jedem Marktsegment vertreten und kann dort seine Uhren verkaufen.
\end{itemize}
	\item Ausbau einer bestehender Produktionsstraße(n)
\begin{itemize}
	\item Mit dem Ausbau einer bestehenden Produktionsstraße ist der Zukauf einer Produktionsstraße mit dem bereits vorhandenen Marktsegment gemeint. Dadurch erzielt das Unternehmen eine Kostensenkung der Produktion für das Marktsegment. 
\end{itemize}
	\item Erweiterung einer Produktionsstraße
\begin{itemize}
	\item Eine bereits gekaufte Produktionsstraße kann ebenfalls erweitert werden. Dabei stehen insgesamt drei Stufen zur Verbesserung bereit. Dadurch wird eine Kostensenkung für die verbesserte Produktionsstraße erzielt.   
\end{itemize}
\end{enumerate}
    
\section{Forschung\&Entwicklung}

\section{Marketing}

\section{Verkauf}

\section{Einkauf}\label{sec:einkauf}
Der Einkauf sollte dem Spieler zunächst drei Optionen bieten:
\begin{enumerate}
\item Der Einkauf von Rohstoffen \par
Da Uhren oft aus sehr individuellen Materialien bestehen, sollte es hier eine Liste möglicher Rohstoffe geben, die je nach Marktsegment ausgewählt werden können.
\begin{itemize}
\item	Holz	\\
\item	Textilien	\\
\item	Kunststoffe	\\
\item	Edelstahl \\
\item	Reintitan \\
\item	Aluminium \\
\item	High-Tech-Keramik \\
\item	Gold, Silber, Platin
\item Edelsteine und Perlen \\
\item Leder, Wildleder, Krokodilsleder \\
\item	Fischhaut
\item	Glas \\
\end{itemize}
\item Der Einkauf von halbfertigen und fertigen Erzeugnissen \par
Da die Hauptbestandteile einer Uhr aus vielen kleinen Einzelteilen bestehen, werden diese meist durch Fremdbezug erworben. Bei dieser Option sollen also die bereits hergestellten Hauptbestandteile eingekauft werden.
\item Die Eigenproduktion durch beispielsweise einen eigenen Holzanbau
\end{enumerate}

Da die zweite Option auch in der Realität weit verbreitet ist, wurde beschlossen, dass sich auf die Auswahl eines Armbands, eines Gehäuses und eines Uhrwerks beschränkt wird.

Des Weiteren sollte es die Möglichkeit geben, Lieferanten gezielt auswählen zu können und dabei auf Faktoren wie Kosten, Regionalität und Transportmittel zu achten.
Diese Option wurde aufgrund des hohen Aufwands jedoch nicht in das Spiel übernommen.

Auch Skonti und Rabatte sollen im Einkauf möglich sein. Sie sind vor allem mengenabhängig und richten sich nicht nach einzelnen Lieferanten.

Für die Produkte, die im Ökosegment angeboten werden können, sollte es beim Einkauf zusätzliche Informationen zu Siegeln wie beispielsweise dem Blauer Engel für Leder und Fairtrade für Gold und Edelsteine geben.
Diese Möglichkeit kommt als Erweiterung des Spiels in Frage, da sie zunächst nicht spielentscheidend oder notwendig für die Funktionalität ist.

