\section{Einkauf}\label{sec:einkauf}
Der Einkauf sollte dem Spieler zunächst drei Optionen bieten:
\begin{enumerate}
\item Der Einkauf von Rohstoffen \par
Da Uhren oft aus sehr individuellen Materialien bestehen, sollte es hier eine Liste möglicher Rohstoffe geben, die je nach Marktsegment ausgewählt werden können.
\begin{itemize}
\item	Holz	\\
\item	Textilien	\\
\item	Kunststoffe	\\
\item	Edelstahl \\
\item	Reintitan \\
\item	Aluminium \\
\item	High-Tech-Keramik \\
\item	Gold, Silber, Platin
\item Edelsteine und Perlen \\
\item Leder, Wildleder, Krokodilsleder \\
\item	Fischhaut
\item	Glas \\
\end{itemize}
\item Der Einkauf von halbfertigen und fertigen Erzeugnissen \par

\item Die Eigenproduktion durch beispielsweise einen eigenen Holzanbau
\end{enumerate}

Da die zweite Option auch in der Realität weit verbreitet ist, wurde beschlossen, dass sich auf die Auswahl eines Armbands, eines Gehäuses und eines Uhrwerks beschränkt wird.

Des Weiteren sollte es die Möglichkeit geben, Lieferanten gezielt auszuwählen und dabei auf Faktoren wie Kosten, Regionalität und Transportmittel achten zu können.

Für die Produkte, die im Ökosegment angeboten werden können, sollte es beim Einkauf zusätzliche Informationen zu Siegeln wie beispielsweise dem Blauer Engel für Leder und Fairtrade für Gold und Edelsteine geben.