\clearpage
\chapter{Architektur}
In diesem Kapitel wird der technische Aufbau des Spiels beschrieben. Hierbei wird auf die verwendete Architektur sowie die zugrunde liegenden Designentscheidungen eingegangen.

\section{Einflussfaktoren für Architektur}
Die Maßgebenden Einflussfaktoren für die Entscheidung der verwendeten Softwarearchitektur, sind neben dem im Team vorhanden Know-How bezüglich Programmiersprachen und Unit-Tests, die eigenen Anforderungen an die optische Repräsentation des Spiels. Nach einer Aufnahme der Kenntnisse der Teammitglieder war schnell ersichtlich, dass die Programmiersprache Java am verbreitetsten ist und sich in Bezug auf Unit-Tests auch hier die größte Übereinstimmung findet.

Der in der Vorlesung zur Fallstudie vorgestellte Tomcat-Server bietet die Möglichkeit, ein in Java geschriebenes Fachkonzept mit einem Userinterface auf HTML-Basis zu verbinden. Da die in Java zur Verfügung stehende SWING-Bibliothek nicht die gewünschten optischen Resultate erzielt, wurde diese schnell ausgeschlossen.

\section{Übersicht über technischen Aufbau}
Aus den zuvor genannten Beweggründen ergab sich schnell der in ABB dargestellte Aufbau der Software. Im ersten Schritt zur Gliederung der Architektur wurde die aus dem Studium bekannte 3-Schichten-Architektur zugrunde gelegt. Die Software wird hierbei in die ebenen Datenhaltung, Fachkonzept sowie Darstellung untergliedert. Bei den vorgegebenen Anforderungen an die Fallstudie, ist eine Datenhaltung als optional angegeben. Zur Reduzierung des Programmieraufwandes wurde auf eine Datenbankanbindung oder sonstige Datenhaltung verzichtet.

Zur Umsetzung des Fachkonzeptes wird auf die Programmiersprache Java gesetzt. Wobei die gesamte Spiellogik so umgesetzt ist, dass hierauf ein beliebiges Userinterface aufgesetzt werden kann, bzw. die Spielkomponenten auch über eine Eingabekonsole aufrufbar sind. Zum Testen des Fachkonzepts kommt jUnit zum Einsatz. Da als Userinterface eine Webseite dienen soll, wird das Fachkonzept nicht in einer lokalen JVM sondern auf einem Webserver ausgeführt. Der zum Einsatz kommende Tomcat-Server kann sowohl den Java-Code des Fachkonzepts ausführen, als auch über die gewünschte Webseite Eingaben des Anwenders entgegen nehmen, bzw. entsprechende Ausgaben zurückgeben. Die dabei zum Einsatz kommenden Techniken werden in den folgenden Kapiteln näher erläutert.

Bei der Darstellungsebene des Spiels werden die gestalterischen Möglichkeiten von HTML, CSS sowie JavaScript genutzt, indem der Tomcat-Server für den Anwender eine entsprechende Webseite als Userinterface zur Verfügung stellt. Die Bedienung des Spiels erfolgt somit für den Anwender in einem Webbrowser.

Folglich ergibt sich, neben einer logischen Trennung von Fachkonzept und Darstellung, eine technische Trennung, in Form einer Client-Server-Architektur. Auf dem Server wird die Spiellogik ausgeführt, eine dynamische Webseite generiert und auf Eingaben des Anwenders gewartet. Im Browser des Clients wird hingegen die dynamisch generierte Webseite mit den aktuellen Werten des Spielstandes angezeigt, als auch eine Eingabemöglichkeit zur Ausführung von gewünschten Spielaktionen geboten.

\section{Java Servlets}
Auf dem bereits erwähnten Tomcat-Server wird eine Instanz der Klasse Servlet ausgeführt. Hierüber wird es ermöglicht, auf Eingaben des Anwenders zu reagieren und entsprechende Rückgaben auszuliefern. Die Kommunikation zwischen Server und Client erfolgt hierbei mittels Http-Requests. In diesem Fall wird die doPost()-Methode der Klasse HttpServlet überschrieben und mit den gewünschten Funktionalitäten versehen.

Der Aufruf der Methode erfolgt von Seiten des Clients durch absenden eines HTML-Formulars. Im Unterkapitel User Interface wird hier näher drauf eingegangen. Innerhalb der doPost()-Methode können über das request-Objekt auf die vom Anwender getätigten Eingabeparameter zugegriffen werden. Aufgrund der Eingaben des Anwenders werden die entsprechenden Methoden der Spiellogik mit den benötigten Übergabeparametern aufgerufen. Bei einem ggf. stattzufindenden Seitenwechsel im Browser werden weiterhin entsprechende Übergabeparameter dem request-Objekt hinzugefügt und die neue Seite unter Mitgabe des Objekts aufgerufen.

Zur weiteren Vereinfachung des Programmieraufwandes wird innerhalb der Servlet-Instanz nicht nur die Kommunikation mit dem Client koordiniert, sondern auch eine Instanz des Spiels erzeugt. Alle innerhalb der doPost()-Methode getätigten aufrufe bezüglich der Spiellogik werden somit an die Instanz des Spiels übergeben. Auch wenn auf abstrakter Ebene das Servlet und die Spiellogik als zwei getrennte Elemente betrachten werden, ist die Spielinstanz real gesehen eine Instanz innerhalb des Servlets.

\section{JavaServer Page}
Wie oben beschriebenen werden in der doPost()-Methode neue Webseiten unter Mitgabe des request-Objekts aufgerufen. Da eine nur mit HTML erstellte Seite mit dem request-Objekt nicht umgehen kann, kommt zur Erstellung der Webseite JavaServer Page bzw. Expression Language zum Einsatz. Der Einsatz beider beschränkt sich jedoch innerhalb dieses Projektes auf die Angabe des Ziels des HTML-Formulars sowie als Platzhalter für Variablen zur dynamischen Erstellung von Seiteninhalten.

Die Webseite wird wie gewohnt mit HTML, CSS und Javascript erstellt, wobei die dynamisch zu erstellenden Inhalte im Quellcode durch Value Expressions ( \$\{ Value \} ) ersetzt werden. Beim Aufruf der Seite innerhalb der doPost()-Methode werden die Value Expressions mit den gleichnamigen Parametern aus dem request-Objekt ersetzt und der so entstehende HTML-Code an den Browser des Clients ausgeliefert. In diesem Projekt werden hierdurch vor allem Textinhalte innerhalb von HTML-Elementen generiert bzw. bestimmte Werte dem Class-Attribut der HTML-Elemente hinzugefügt.

\section{User Interface}
\section{Code Organisation}

\clearpage
\chapter{Entwickungsumgebung}
\section{Eclipse JAVA EE IDE}
Eclipse ist eine Programmierumgebung als Entwicklungswerkzeug von Software und Programmen aller Art. Es gibt eine Vielzahl von kommerziellen und kostenlosen Erweiterungen. Unter Eclipse ist es möglich nahezu jede Programmiersprache zu entwickeln. Für das Projekt wurde lediglich die Programmiersprache Java für die Logik der Simulation verwendet.
\section{Git}
Bei Git handelt es sich um eine freie Software zur Versionsverwaltung. Git ist Open Source und lediglich per Kommandozeile verfügbar. Dennoch gibt es bereits unzählige grafische UIs, die Git vereinfachen und seine Befehle grafisch darstellen (beispielhaft dafür: Sourcetree, GitHub Desktop oder TortoiseGit) . Git wird auch von zahlreichen Großprojekten verwendet, darunter Android, LibreOffice oder jQuery.
\section{GitHub}
GitHub ist der webbasierte Online-Dienst, welcher Entwicklungsprojekte auf einem Server bereitstellt und somit Filehosting betreibt. Grundsätzlich ist es nicht anderes als Git, eben nur nicht lokal, zudem bringt GitHub seinen eigenen grafischen UI Client namens GitHub Desktop mit. Ein grafischer Client wurde in diesem Projekt nicht mitgliederübergreifend eingesetzt.
\section{LaTeX}
Für die Erstellung der Dokumentation wurde das Softwarepaket LaTeX genutzt. Als Editor wurde TeXstudio verwendet.\\
LaTeX ist ein Softwarepaket, welches die Benutzung des Textsatzsystems TeX mit Hilfe von Makros vereinfacht.
LaTeX ist die populärste Methode TeX zu verwenden.
TeXstudio ist ein plattformunabhängiger Editor zur Erstellung von genannten LaTeX Dokumenten.\\
Beide sind Open-Source und kostenlos. Neben TeXstudio gibt es noch eine Vielzahl an weiteren Editoren, speziell für LaTeX Dokumente.

