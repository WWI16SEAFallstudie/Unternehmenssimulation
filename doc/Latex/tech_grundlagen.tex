\clearpage
\chapter{Architektur}
\section{Übersicht über technischen Aufbau}
\section{Java Servlets}
\section{JavaServer Page}
\section{User Interface}
\section{Code Organisation}

\clearpage
\chapter{Entwickungsumgebung}
\section{Eclipse JAVA EE IDE}
Eclipse ist eine Programmierumgebung als Entwicklungswerkzeug von Software und Programmen aller Art. Es gibt eine Vielzahl von kommerziellen und kostenlosen Erweiterungen. Unter Eclipse ist es möglich nahezu jede Programmiersprache zu entwickeln. Für das Projekt wurde lediglich die Programmiersprache Java für die Logik der Simulation verwendet.
\section{Git}
Bei Git handelt es sich um eine freie Software zur Versionsverwaltung. Git ist Open Source und lediglich per Kommandozeile verfügbar. Dennoch gibt es bereits unzählige grafische UIs, die Git vereinfachen und seine Befehle grafisch darstellen (beispielhaft dafür: Sourcetree, GitHub Desktop oder TortoiseGit) . Git wird auch von zahlreichen Großprojekten verwendet, darunter Android, LibreOffice oder jQuery.
\section{GitHub}
GitHub ist der webbasierte Online-Dienst, welcher Entwicklungsprojekte auf einem Server bereitstellt und somit Filehosting betreibt. Grundsätzlich ist es nicht anderes als Git, eben nur nicht lokal, zudem bringt GitHub seinen eigenen grafischen UI Client namens GitHub Desktop mit. Ein grafischer Client wurde in diesem Projekt nicht mitgliederübergreifend eingesetzt.
\section{LaTeX}
Für die Erstellung der Dokumentation wurde das Softwarepaket LaTeX genutzt. Als Editor wurde TeXstudio verwendet.\\
LaTeX ist ein Softwarepaket, welches die Benutzung des Textsatzsystems TeX mit Hilfe von Makros vereinfacht.
LaTeX ist die populärste Methode TeX zu verwenden.
TeXstudio ist ein plattformunabhängiger Editor zur Erstellung von genannten LaTeX Dokumenten.\\
Beide sind Open-Source und kostenlos. Neben TeXstudio gibt es noch eine Vielzahl an weiteren Editoren, speziell für LaTeX Dokumente.

