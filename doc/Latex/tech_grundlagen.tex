\clearpage
\chapter{Architektur}
\section{Übersicht über technischen Aufbau}
\section{Java Servlets}
\section{JavaServer Page}
\section{User Interface}
\section{Code Organisation}

\chapter{Entwickungsumgebung}
\section{Eclipse JAVA EE IDE}
\section{Git}
\section{GitHub}
\section{LaTeX}
Für die Erstellung der Dokumentation wurde das Softwarepaket LaTeX genutzt. Als Editor wurde TeXstudio verwendet.\\
LaTeX ist ein Softwarepaket, welches die Benutzung des Textsatzsystems TeX mit Hilfe von Makros vereinfacht.
LaTeX ist die populärste Methode TeX zu verwenden.
TeXstudio ist ein plattformunabhängiger Editor zur Erstellung von genannten LaTeX Dokumenten.\\
Beide sind Open-Source und kostenlos. Neben TeXstudio gibt es noch eine Vielzahl an weiteren Editoren, speziell für LaTeX Dokumente. 

