\clearpage
\chapter{Einleitung}
\begin{tabbing}
Vorlesung
\hspace{0.2cm}\=\kill
Vorlesung: \> Systemanalyse - Fallstudie \\
Titel: \> Watch Tycoon 2017 
\end{tabbing}
In dieser Arbeit wird die Erarbeitung einer Konzeption und Umsetzung eines Unternehmensplanspiels
im Rahmen der Fallstudie beschrieben. Das Ziel dieser Arbeit ist die Entwicklung eines computergestützten Planspiels, auf dem die Spieler in Konkurrenz zueinander stehen. Der Entwurf und die Entwicklung des Spiels bilden die Hauptaspekte dieser Arbeit. Die Entwicklung wird in der Programmiersprache Java realisiert. Betriebswirtschaftliche Aspekte werden in dem Spiel
ebenfalls berücksichtigt. Dazu zählen zum Beispiel: Marktveränderungen, variable Rohstoffpreise, Betriebskosten oder Verkaufspreise. Des Weiteren wird die Projektorganisation vorgestellt, welche während dem Projekt eine wichtig Rolle gespielt hat.
Im weiteren Verlauf wird die Entwicklung des User Interfaces erläutert. Dazu wird ein Mock-Up mit Hilfe von HTML und Bootstrap angefertigt und später zu einem fertigen Layout entwickelt. Mit einem Tomcat-Webserver wird dann das Spiel ablaufbereit gemacht. Zuletzt werden theoretische Verbesserungsmöglichkeiten aufgezeigt und ein Fazit über das gesamte Projekt gezogen.   

