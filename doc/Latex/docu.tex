\clearpage
\chapter{Einleitung}
\section{Aufbau der Arbeit}
%\begin{tabbing}
%Vorlesung
%\hspace{0.2cm}\=\kill
%Vorlesung: \> Systemanalyse - Fallstudie \\
%Titel: \> Watch Tycoon 2017 
%\end{tabbing}
In dieser Arbeit wird die Erarbeitung einer Konzeption und Umsetzung eines Unternehmensplanspiels im Rahmen der Fallstudie beschrieben. Das Ziel dieser Arbeit ist die Entwicklung eines computergestützten Planspiels, auf dem die Spieler in Konkurrenz zueinander stehen und ein fiktiver Markt gebildet wird. Der Entwurf und die Entwicklung des Spiels bilden die Hauptaspekte dieser Arbeit. \\
Die Entwicklung wird in der Programmiersprache Java realisiert. Funktionen wurden zusätzlich in JavaScript geschrieben. Mit einem Tomcat-Webserver wird dann das Spiel lauffähig gemacht. \\
Betriebswirtschaftliche Aspekte werden in dem Spiel ebenfalls berücksichtigt. Dazu zählen zum Beispiel: variable Nachfrage der einzelnen Produkte, Betriebskosten oder Verkaufspreise. Die Marktveränderungen werden mittels eigen entwickeltem Algorithmus berechnet. \\
Die Entwicklung des grafischen User Interfaces wurde mit Hilfe von Bootstrap und HTML realisiert. Das Mock-Up, was ebenfalls in HTML statisch erstellt wurde, war dabei die Vorlage. \\ 
Mittels JUnit-Test werden die Szenario-Test durchgeführt und die Code-Coverage überprüft. Ziel hierbei ist eine Coverage > 65\%. Zudem werden die Use-Cases definiert und erläutert. \\
Des Weiteren wird die Projektorganisation vorgestellt, welche während dem Projekt eine wichtig Rolle gespielt hat. Hierbei wurde zusätzlich ein Meilenstein-Diagramm in \ref{fig:abb} auf Seite \pageref{sec:meilenstein} verwendet um die einzelnen Meilensteine optisch darzustellen.\\
Zuletzt werden theoretische Verbesserungsmöglichkeiten aufgezeigt und ein Fazit über das gesamte Projekt gezogen.

\clearpage
\section{Herangehensweise}\label{sec:herangehensweise}
Das Team setzt sich aus sieben Mitgliedern zusammen. Zu Beginn wurde in einem Kick-Off Meeting das Team bzw. die Industrie ausgewählt, aus welcher das Planspiel bestehen soll. Nach kurzer Überlegung konnte schnell in der Uhrenindustrie ein Nenner gefunden werden. \\
Auch bei der recht großen Anzahl an Teammitglieder konnte die Aufgaben sinnvoll aufgeteilt werden. So herrschte eine recht ausgewogene Balance, was die verschiedenen Schwerpunkte und Erfahrungen der Mitglieder betraf. Im Laufe des Projekte zeigte sich auch hierbei der Vorteil eines breiten technischen Portfolios. Dennoch konnten einige Problemstellung nicht mit dem reinen Wissen aus dem Studium abgedeckt werden. Ebenso waren Kenntnisse in der Mathematik und Wirtschaftswissenschaft gefragt um beispielsweise einen fiktiven Markt mit einem realistischen Algorithmus zu versehen.      

\section{Grundgedanke des Spiels}
Das Spiel \enquote{Watch Tycoon 2017} soll in einem rundenbasierten computergestützten Planspiel mit bis zu vier Spielern eine nahezu reale Unternehmenssimulation in einem Industriezweig darstellen. Der hierfür verwendet Industriezweig ist die Uhrenindustrie.\\
Der Spieler hat die Abteilungen Produktion, F\&E, Einkauf, Verkauf und Marketing zur Verfügung um eine bestmögliche Strategie auszuarbeiten und ein breites Spektrum an Varianz zu erhalten.\\
Der fiktive Markt soll mittels einem mathematischen Algorithmus so real wie möglich erstellt werden und den Mittelpunkt für Angebot und Nachfrage bilden.













