%
%   Prof. Dr. Julian Reichwald
%   auf Basis einer Vorlage von Prof. Dr. Jörg Baumgart
%   DHBW Mannheim
%
%
%	ACHTUNG: Für das Erstellen des Literaturverzeichnisses wird das modernere Paket biblatex
%			 in Kombination mit biber verwendet -- nicht mehr das ältere BibTex!
% 			 Bitte stellen Sie ggf. Ihre TeX-Umgebung
% 			 entsprechend ein (z.B. TeXStudio: Einstellungen --> Erzeugen --> Standard Bibliographieprogramm: biber)
%

\documentclass[
	12pt,
	BCOR=5mm,
	DIV=12,
	headinclude=on,
	footinclude=off,
	parskip=half,
	bibliography=totoc,
	listof=entryprefix,
	toc=listof,
	pointlessnumbers,
	plainfootsepline]{scrreprt}

%	Konfigurationsdatei einziehen
\input{config}

\begin{document}

%% BITTE GEBEN SIE HIER DEN TITEL UND DIE AUTORIN / DEN AUTOR DER ARBEIT AN!
\TitelDerArbeit{Watch Tycoon 2017}
%\AutorDerArbeit{Erika Musterfrau}

\begin{titlepage}
\begin{minipage}{\textwidth}
		\vspace{-2cm}
		\noindent  
		\centering \includegraphics{img/logo.jpg}
\end{minipage}
\vspace{1em}
\sffamily
\begin{center}
	\textsf{\large{}Duale Hochschule Baden-W\"urttemberg\\[1.5mm] Mannheim}\\[2em]
	\textsf{\textbf{\Large{}Systemanalyse - Fallstudie}}\\[3mm]
	\textsf{\textbf{\DerTitelDerArbeit}} \\[1.5cm]
	\textsf{\textbf{\Large{}Studiengang Wirtschaftsinformatik}\\[3mm] \textsf{Studienrichtung Software Engineering}}
	
	\vspace{5em}

\begin{minipage}{\textwidth}

\begin{tabbing}
	Wissenschaftlicher Betreuer:
	\hspace{0.5cm}\=\kill
	Kurs: \> WWI 16 SEA \\[1.5mm]
	Studiengangsleiter: \> Prof. Dr. Julian Reichwald  \\[1.5mm]
	Dozent: \> Hr. Gregor Tielsch  \\[1.5mm]
	Teammitglieder: \> Luisa Karl, \\ \> Rebekka Henn, \\ \> Miriam Wolf, \\ \> Ewald Anselm, \\ \> Tillmann Heß, \\ \> Erik Schmitt, \\ \> Nico Feil \\[1.5mm]
	Bearbeitungszeitraum: \> 28.08.2017 -- 13.11.2017
\end{tabbing}
\end{minipage}

\end{center}

\end{titlepage}

\pagenumbering{roman} % Römische Seitennummerierung
\normalfont

%--------------------------------
% Verzeichnisse - nicht benötige Verzeichnisse bitte auskommentieren / löschen.
%--------------------------------

%   Sperrvermerk
%\input{nondisclosurenotice}

%	Kurzfassung
%\input{abstract}

%	Inhaltsverzeichnis
\tableofcontents

%	Abbildungsverzeichnis
%\listoffigures

%	Tabellenverzeichnis
%\listoftables

%	Listingsverzeichnis
%\lstlistoflistings

% 	Algorithmenverzeichnis
% \listofalgorithms

% 	Abkürzungsverzeichnis (siehe Datei acronyms.tex!)
%\input{acronyms}

%--------------------------------
% Start des Textteils der Arbeit
%--------------------------------
\clearpage
\ihead{\chaptername~\thechapter} % Neue Header-Definition
\pagenumbering{arabic}  % Arabische Seitenzahlen

% 	Einleitung
\clearpage
\chapter{Einleitung}
\begin{tabbing}
Vorlesung
\hspace{0.2cm}\=\kill
Vorlesung: \> Systemanalyse - Fallstudie \\
Titel: \> Watch Tycoon 2017 
\end{tabbing}
In dieser Arbeit wird die Erarbeitung einer Konzeption und Umsetzung eines Unternehmensplanspiels
im Rahmen der Fallstudie beschrieben. Das Ziel dieser Arbeit ist die Entwicklung eines computergestützten Planspiels, auf dem die Spieler in Konkurrenz zueinander stehen. Der Entwurf und die Entwicklung des Spiels bilden die Hauptaspekte dieser Arbeit. Die Entwicklung wird in der Programmiersprache Java realisiert. Betriebswirtschaftliche Aspekte werden in dem Spiel
ebenfalls berücksichtigt. Dazu zählen zum Beispiel: Marktveränderungen, variable Rohstoffpreise, Betriebskosten oder Verkaufspreise. Des Weiteren wird die Projektorganisation vorgestellt, welche während dem Projekt eine wichtig Rolle gespielt hat.
Im weiteren Verlauf wird die Entwicklung des User Interfaces erläutert. Dazu wird ein Mock-Up mit Hilfe von HTML und Bootstrap angefertigt und später zu einem fertigen Layout entwickelt. Mit einem Tomcat-Webserver wird dann das Spiel ablaufbereit gemacht. Zuletzt werden theoretische Verbesserungsmöglichkeiten aufgezeigt und ein Fazit über das gesamte Projekt gezogen.   



% 	Grundgedanke des Spiels
\clearpage
\chapter{Grundgedanke des Spiels}
Das Spiel "Watch Tycoon 2017" soll in einem rundenbasierten computergestützten Planspiel mit bis zu vier Spielern eine nahezu reale Unternehmenssimulation in einem Industriezweig darstellen. Der hierfür verwendet Industriezweig ist die Uhrenindustrie.\\
Der Spieler hat die Abteilungen Produktion, F\&E, Einkauf, Verkauf und Marketing zur Verfügung um eine bestmögliche Strategie auszuarbeiten und ein breites Spektrum an Varianz zu haben.\\



% 	Industriezweig Uhr
\clearpage
\chapter{Der Industriezweig Uhr}
\section{Geschichtliches zur Uhrindustrie}
Die Schweiz ist derzeit das Land mit den meisten Uhrenexporten. Gemessen an den Exporten ist die Uhrenindustrie innerhalb der Schweiz jedoch nur die drittgrößte Industrie. Weltweit gefolgt wird die Schweiz Hongkong und China.\\ 
In den 1970er und 1980er Jahren erhielt die klassische schweizer Uhrenindustrie Konkurrenz durch die Entstehung der elektrischen Uhren und den asiatischen Markt. Nachdem sie sich bis 2015 wieder etwas erholen konnte und die Exporte von 4,3 Milliarden Franken im Jahr 1986 auf 21,5 Milliarden im Jahr 2015.\\ 
Mittlerweile schaffen es vor allem die Großmarken, sich gegen die neue Konkurrenz der Smart Watches durchzusetzen, indem sie entweder hochwertige Luxusuhren herstellen, die als Schmuckstück und Statussymbol dienen oder indem sie selbst Smart Watches entwickeln.

\section{Marktsituation}
Der reale Uhrenmarkt wird (im Gegensatz zu unserem fiktiven Uhrenmarkt) von nur wenigen Ländern beherrscht. Hong Kong, China und die Schweiz sind dabei an der Spitze. Hong Kong und China sind, wie in vielen anderen Brachen, mit Massenproduktionen die weltweit größten Uhrenproduzent. Die Uhren sind im Niedrigpreissegment angesiedelt. Hingegen ist die Schweiz mit ihren berühmten \enquote{Schweizer Uhrwerken} Weltmarktführer im hochwertigen Marktbereich. Aufgrund der hochpreisigen Uhren macht die Schweiz nur einen verschwindend geringen Anteil an der globalen Produktion aus, hingegen sie wertemäßig mit Abstand das führende Exportland ist. \\   
Die Firmen, die symbolisch für die Weltmarktführerschaft stehen, sind Swatch, Richemont und Rolex.\\
Zusammen machen die drei Marken 40\% bis 50\% des weltweiten Uhrenumsatzes aus. Im Jahr 2016 exportierte die Schweiz Uhren im Wert von 19,8 Milliarden EURO. Gefolgt von Hong Kong (8,8) und China (5,3)




% 	Abteilungen
\clearpage
\chapter{Abteilungen}
\section{Produktion}


% 	Spielplan
\clearpage
\chapter{Spielplan}
Blabla
\includepdf{img/Spielplan.pdf}


% 	Architektur/tech. Grundlagen
\clearpage
\chapter{Architektur/technische Grundlagen}
\section{Verwendete Programme, Software \& Programmiersprachen}
\subsection{Programmiersprachen}
\subsection{Software}
\subsection{Programme}
\subsubsection{Git}
\subsubsection{Tomcat}
\subsubsection{Eclipse}
\subsubsection{HTML}
\subsubsection{Java}
\subsubsection{Bootstrap}
\subsubsection{LaTex}

% 	Projektorganisation
\clearpage
\chapter{Projektorganisation}
\section{Aufgabenverteilung}
\begin{enumerate}
	\item Coding
	\begin{itemize}
		\item Miriam Wolf, Erik Schmidt, Ewald Anselm
	\end{itemize} 
	\item JUnit-Tests: Erstellung, Realisierung und Implementierung
	\begin{itemize}
		\item Rebekka Henn, Luisa Karl
	\end{itemize} 
	\item Coding Layout: Erstellung, Realisierung und Implementierung
	\begin{itemize}
		\item Ewald Anselm, Erik Schmidt
	\end{itemize} 
	\item Markt Algorithmen: Erstellung, Realisierung und Implementierung
	\begin{itemize}
		\item Tilman Heß
	\end{itemize} 	
	\item Dokumentation und Präsentation
	\begin{itemize}
	\item Nico Feil
	\end{itemize}
\end{enumerate}
\clearpage
\section{Meilensteine}\label{meilenstein}
\begin{figure}[!h]
	\centering
	\includegraphics[angle=90, scale=0.47]{img/Meilensteine_Fallstudie.pdf}
	\label{fig:abb}
	\caption{Meilenstein-Diagramm} 
\end{figure}


% 	Ausbaumöglichkeiten
\clearpage
\chapter{Verbesserungsmöglichkeiten}
Während der Durchführungsphase zu diesem Projekt sind einige Verbesserungsmöglichkeiten entstanden, welche aufgrund der Zeit aber nicht mehr realisiert werden konnten. Nachfolgenden finden Sie alle Verbesserungsmöglichkeiten und eine kurze Begründung.
\begin{enumerate}
	\item Erweiterung des Unternehmens um einen HR-Bereich
\begin{itemize}
	\item
\end{itemize} 
	\item Kreditaufnahme/-tilgung im Einkauf-Bereich
\begin{itemize}
	\item
\end{itemize} 
	\item Speicherung des Spielstands
\begin{itemize}
	\item
\end{itemize} 
	\item Anfälligkeit/Defekte der Produktionsstraßen
\begin{itemize}
	\item
\end{itemize} 	
\end{enumerate}






% 	Fazit/Ausblick
\clearpage
\chapter{Fazit/Ausblick}
Grundsätzlich ist die Aufgabe eine nette und fordernde Abwechslung zum normalen DHBW-Alltag. Das Planen und Entwickeln der verschiedenen Bestandteile hat Spaß gemacht und jeden in seiner Entwicklung und auf seine Weise weitergebracht.\\
Wir konnten uns recht schnell auf unsere "Industrie" einigen und hatten so mehr Zeit für die Planung und Entwicklung. Die meiste Zeit wurde für die Konzeption und Entwicklung der einzelnen Bestandteile verwendet. Es gab einige Diskussionen rund um die einzelnen Unternehmensabteilungen und deren Methoden. Grundsätzlich kamen wir aber immer auf einen Nenner im Sinne der Aufgabe.\\
Negativ war leider die kurze Zeit, die wir insgesamt hatten. Einige Ideen konnten so nur teilweise oder gar nicht umgesetzt werden. Auch das Schreiben einer recht ausführlichen Dokumentation hat uns einige Zeit gekostet zumal die Klausuren für dieses Semester ebenfalls an standen.\\   
Zusammenfassend kann man sagen, dass die Fallstudie ein guter Ansatz für einen realen Bezug ist, aber der enge Zeitraum einiges verhindert und somit rückblickend eher negativ bewertet werden muss.   

%	Literaturverzeichnis
\ihead{} % Neue Header-Definition
\printbibliography[title=Literaturverzeichnis]
\cleardoublepage

% Der Anhang beginnt hier - jedes Kapitel wird alphabetisch aufgezählt. (Anhang A, B usw.)
\appendix
\ihead{\appendixname~\thechapter} % Neue Header-Definition

% appendix.tex einziehen
\input{appendix}

% Ehrenwörtliche Erklärung ewerkl.tex einziehen
%\input{ewerkl.tex}


\end{document}
